\chapter{SIR\label{chapter:sir}}
\lhead{SIR}
\begin{refsection}
\chapterauthor{Max Obrist und Martin Stypinski}

Die \emph{Epidemiologie} ist eine wissenschaftliche Disziplin, welche sich mit Ursache, Verbreitung und Folgen von gesundheitsbezogenen Ereignissen in einer Population befasst.
Im Gegensatz zur klassischen Medizin befasst sich die Epidemiologie nicht damit, wie eine Krankheit geheilt werden kann, oder eine kranke Person geheilt werden kann, sondern damit, wie sich eine spezielle Krankheit ausbreitet, und mit welchen Mitteln die Krankheit in der Gesamtbevölkerung besiegen kann.

Die \emph{mathematische Epidemiologie} -- womit wir uns in diesem Kapitel ansatzweise befassen -- ist ein Teilgebiet der Epidemiologie sowie der \emph{theoretischen Biologie}.
Die mathematische Epidemiologie befasst sich speziell mit formalen Modellen zur Ausbreitung von Krankheiten, stellt also Fragen nach z.B. der Form oder der Ausbreitungsgeschwindigkeit von ansteckbaren Krankheiten, beispielsweise von Influenza, Masern oder auch Ebola. 
Das \emph{SIR-Model}, welches wir auf den nächsten Seiten detailliert anschauen werden, ist ein mathematisches Modell, mit welchem der Verlauf einer Krankheit in einer Population modelliert werden kann.

\section{Problemstellung}
Wir stellen uns vor, dass in einer Bevölkerungsgruppe eine Krankheit ausbricht. 
Es handelt sich dabei um einen neuen Influenza-Stamm handeln.
Bei einer Grippe wird ein Träger, nachdem er die Krankheit ausgestanden hat, bekanntermassen immun gegen diesen speziellen Stamm (Aber nicht gegen andere Stämme von Influenza).
Nachdem die Krankheit bereits ein wenig durch die Bevölkerung gegangen ist, lassen sich anhand der ersten Daten einige Informationen über die Krankheit erfassen.
Zum Beispiel kann herausgefunden, wie Infektiös eine Krankheit ist, oder wie lange es nach einer Infektion durchschnittliche dauert, bis ein Träger die Infektion überstanden hat.

Anhand dieser Informationen soll nun ein Modell entwickelt werden, mit welchem gewisse Vorhersagen über den Verlauf der Krankheit getroffen werden können. 
Es soll nochmals wiederholt werden, dass mit dem Verlauf der Krankheit nicht der Verlauf der Krankheit in einem Individuum gemeint ist, sondern wie sich die Krankheit in der Bevölkerung selbst ausbreitet. 
Der Patient ist sozusagen also nicht ein einzelner Kranker, sondern eine ganze Population aufs mal.

Der Einfachheit halber nehmen wir an, dass die Grösse der Population konstant ist.
Die Geburten- und Sterberate in der Bevölkerung werden damit ignoriert.

\section{SIR-Modell}
Bei allen Krankheiten, welche sich wie die oben beschriebene Grippe verhalten -- Ein gesunden Subjekt wird infiziert, ist eine gewisse Zeit lang Krank und ist nach der Genesung immun gegen den Erreger -- lässt sich die Bevölkerung in 3 Kompartimente aufteilen. 
\begin{description}
  \item [Susceptibles ($S$)] sind Individuen, welche von einem Träger der Krankheit angesteckt werden können. Sie Tragen die Krankheit weder in sich, noch sind sie bereits immun gegen die Krankheit. Werden sie angesteckt, werden sie zu \emph{Infected}.
  \item [Infected ($I$)] sind Personen, die zum aktuellen Zeitpunkt mit der Krankheit infiziert sind, und auch andere Individuen infizieren können. Nach einer gewissen Zeit genesen sie, und verschieben sich dann in das Kompartiment \emph{Removed}.
  \item [Removed ($R$)] sind jene Individuen, welche Immun gegen die Krankheit sind und nicht erneut angesteckt werden können. Im normalen SIR-Modell wird dabei nicht zwischen genesenen und gestorbenen Individuen unterschieden, weshalb dieses Kompartiment schlicht \emph{Removed} genannt wird.
\end{description}

Es wurde bereits erwähnt, dass davon ausgegangen wird, dass die Bevölerung $N$ konstant ist, also $S + I + R = const$. Damit muss die Summe aller Ableitungen konstant sein, wie
\begin{align*}
  \frac{d}{dt}\left(S+I+R\right) = 0
\end{align*}
aufzeigt. Das SIR-Modell beschreibt nun, wie genau sich eine Krankheit in der Bevölkerung ausbreitet, wie sich also die Individuen durch diese 3 Kompartimente bewegen.

\textbf{TODO: Grafik SIR Modell}

Um dies zu beschreiben, muss bekannt sein, mit welchen Raten sich die Individuen zwischen den Kompartimenten verschieben.

Betrachten wir zuerst den Übergang von $S$ nach $R$.
Dieser Übergang ist davon Abhängig, wie oft ein infiziertes Individuum ein anderes anstecken kann.
Dies ist natürlich einerseits davon abhängig, wie infektiös eine Krankheit ist, diese Infektionsrate wird $\beta$ genannt.
Gleichzeit ist dieser Übergang davon abhängig, wie viele Individuen überhaupt angesteckt werden können. 
Pro Zeiteinheit sind dies $\beta N$ Individuen.
Nun ist aber nicht jedes Individuum der Population $N$ tatsächlich \emph{Susceptible}, die Wahrscheinlichkeit ein solches Individuum zu treffen ist $\frac{S}{N}$.
Somit kann ein einzelnes Individuum pro Zeiteinheit 
\begin{align*}
  \beta S N \frac{S}{N} = \beta S
\end{align*}
andere Individuen infizieren.
Alle \emph{Infected} zusammen können damit pro Zeiteinheit $\beta S I$ Individuen infizieren. 
$\beta$ ist also nicht nur davon abhängig, wie ansteckbar eine Krankheit ist, sondern auch wie wahrscheinlich der Kontakt mit einem infizierten Individuum ist. 
Es gibt damit einfach Massnahmen, um diesen Faktor zu beeinflussen, zum Beispiel eine gute Hygiene oder auch Quarantänemassnahmen.

Der Übergang von $I$ nach $R$ ist etwas einfacher zu erklären. 
Nach einer gewissen Zeit von durchschnittlich $k$ Zeiteinheiten genest ein \emph{Infected} Individuum wieder.
Der Kehrwert dieser Dauer wird Genesungsrate genannt und $\gamma$ genannt, also $\gamma = \frac{1}{k}$.
Von allen \emph{Infected} wird damit pro Zeiteinheit $\gamma I$ ins Kompartiment $R$ verschoben.
Auch $\gamma$ kann mit bestimmten Massnahmen beeinflusst werden, z.B. indem Medikamente an die Bevölkerung verteilt werden.

Damit sind alle Informationen bekannt, welche benötigt werden, um unsere Differentialgleichungen aufzustellen. 
\begin{alignat*}{3}
  \frac{dS}{dt} & = - && \beta S I  \\
  \frac{dI}{dt} & =   && \beta S I - && \gamma I \\
  \frac{dR}{dt} & =   &&             && \gamma I 
\end{alignat*}
Die 3 Gleichungen sind dabei einfach zu erklären. 
Aus dem Kompartiment $S$ verschiebt sich jede Zeiteinheit eine gewisse Anzahl Individuen nach $I$, wobei dieser Wert abhängig ist vom Faktor $\beta$ sowie der Anzahl Individuen in den Kompartimenten $S$ und $I$.
Diese tauchen dann im Kompartiment $I$ auf, aus welchem gleichzeitig jede Zeiteinheit eine gewisse Zahl von Individuen nach $R$ verschiebt, ein Wert der Abhängig ist vom $\gamma$ sowie der Anzahl der Individuen in Kompartiment $I$. 
Damit haben wir die Grundlagen des SIR-Modells hergeleitet. 

Wenn die Parameter $\beta$ und $\gamma$ bekannt sind, kann damit ein Model erstellt werden, mit welchem die Ausbreitung einer Krankheit in der Bevölkerung modelliert werden kann.

\textbf{TODO: Grafik SIR Modell mit speziellen Parametern}

Mit diesem Modell lassen sich somit diverse Charakteristiken einer Krankheit erkennen. 
Somit kann nun bestimmt werden, zu welchem Zeitpunkt eine Krankheit am schwersten wütet. 
Aber es gibt noch weitere Informationen, welche anhand des SIR-Modells gefunden werden können.
Ein speziell wichtiger Faktor, der anhand dieses Modells bestimmt werden kann, ist die Impfquote, was wir im folgenden genauer anschauen möchten.

Um eine Aussage über die Impfquote zu treffen, reicht es, die verschiedenen Gleichungen genauer zu betrachten, eine nummerische Lösung ist dafür nicht vonnöten.
Wir beginnen mit der einfachen Gleichung
\begin{align*}
  \frac{dR}{dt} &= \gamma I,
\end{align*}
welche ja gar keine DGL beschreibt. 
Es ist bekannt, dass die Gesamtpopulation $N$ konstant ist. 
Ausserdem ist bekannt, dass alle Kompartimente zusammen die Gesamtpopulation ergeben müssen, also
\begin{align*}
  S + I + R = N.
\end{align*}
Durch Umformung der Gleichung kann $R$ basierend auf den anderen Kompartimenten dargestellt werden als
\begin{align*}
  R = N - S - I.
\end{align*}
Damit kann $R$ aus dem Spiel genommen werden und $S$ sowie $I$ als 2-Dimensionales System dargestellt werden.

Betrachten wir nun die weiteren Gleichungen, beginnend mit 
\begin{align*}
  \frac{dS}{dt} & = -\beta S I.
\end{align*}
Da eine negative Population wenig Sinn macht, müssen $S$ und $I$ immer positiv sein. 
Dasselbe gilt auch für den Parameter $\beta$, welcher ja die Rate darstellt, mit welcher Individuen vom Kompartiment $S$ nach $I$ verschoben werden.
Für das Gesamtsystem $-\beta S I$ bedeutet dies, dass die Ableitung von $S$ zu jedem Zeitpunk kleiner oder gleich 0 ist. $S$ kann damit immer nur abnehmen, aber zu keinem Zeitpunkt wieder zunehmen (wie erwähnt wird die Geburtenrate ignoriert).

Interessant ist nun vor allem die DGL
\begin{align*}
  \frac{dI}{dt} & = \beta S I - \gamma I,
\end{align*}
welche Beschreibt, wie sich das Kompartiment $I$ verhält, also wie gross der Anteil der \emph{Infected} in der Gesamtpopulation ist.
Dafür betrachten wir die einzelnen Bestandteile der DGL separat. 
Im Umkehrschluss zur gerade betrachteten DGL für das Kompartiment $S$ bedeutet $\beta S I$, dass der Anteil der \emph{Infected} von Seiten des Kompartiments $S$ immer zunimmt.
Der zweite Term $- \gamma I$ wiederum ist immer negativ. 
Die Population in $I$ kann ebenfalls nicht negativ sein, und der Parameter $\gamma$ beschreibt die Genesungsrate, welche wie $\beta$ immer positiv ist.
Es besteht also sowohl ein stetiger Anstieg von Individuen aus dem Kompartiment $S$, genauso wie ein stetiger Abgang von Individuen nach $R$.

Daraus lässt sich ableiten, dass die Anzahl infizierter Individuen solange zunimmt, wie $\beta S I \ge \gamma I$. Sobald das Verhältnis kippt, wandern die infizierten Individuen schneller ins Kompartiment $R$ ab, als neue Individuen infiziert werden.
Dieses Maximum befindet sich dort, wo die Ableitung von $I$ gleich null ist, also
\begin{align*}
  \frac{dI}{dt} = 0.
\end{align*}
Um dies aufzulösen wird DGL 
\begin{align*}
  \frac{dI}{dt} = \beta S I - \gamma I
\end{align*}
umformliert, indem $I$ ausgeklammert wird, also
\begin{align*}
  \frac{dI}{dt} = \left(\beta S - \gamma \right) I.
\end{align*}
Da der Wert von I nicht bestimmt werden kann und nicht klar ist, wann dieser 0 ist, setzen wir $\left(\beta S - \gamma \right)$ und lösen nach $S$ auf:
\begin{align*}
  \beta S - \gamma &= 0 \\
  S &= \frac{\gamma}{\beta}
\end{align*}

Dieser Punkt $\frac{\gamma}{\beta}$ ist die Nullkline des Systems. 

\textbf{TODO: Grafik Nullkline}

Sobald also der Anteil von $S$ an der Gesamtpopulation $N$ einen kleineren Anteil als $\frac{\gamma}{\beta}$ hat, ist der Verlauf der Krankheit abnehmend und der Anteil der \emph{Infected} nimmt immer weiter ab.
Interessant ist dabei vor allem, dass dieser Punkt einzig und alleine davon abhängig ist, wie gross der Anteil der \emph{Susceptibles} in der Population ist. 
Der Anteil der \emph{Infected} oder \emph{Removed} ist für diese Aussage nicht relevant. 

Diese Erkentnis lässt sich nun dazu nutzen, eine Impfquote zu berechnen. 
Was eine Impfung bewirkt, ist ja grundsätzlich nichts anderes, als dass ein Individuum ohne den Umweg über das Kompartiment $I$ zu machen, ins Kompartiment $R$ verschoben wird und die Startbedingungen des Systems geändert werden.
Wenn also ein Anteil von mehr als $\frac{\gamma}{\beta}$ in der Population von Beginn weg im Kompartiment $R$ ist -- im Umkehrschluss bedeutet dies, ein Anteil weniger als $\frac{\gamma}{\beta}$ ist im Kompartiment $S$ -- kann eine alfällige Infektion sich nicht ausbreiten.
Natürlich wird auch in dieser Situation eine Individuen aus $S$ mit der Krankheit angesteckt, eine Infektion ist natürlich immer möglich. 
Allerdings nimmt der Anteil der \emph{Infected} immer mehr ab, es kann also nicht zu einer grossangelegten Epidemie kommen.

\section{Varianten des SIR-Models}
\textbf{TODO: Dinge wie SIS, SEIR Model erläutern. Allenfalls weitere Faktoren anschauen wie Geburten- und Sterberate oder so...}

\section{Ebola}
\textbf{TODO: Real World Beispiel Ebola}

\section{Zombies}

\section{Weitere Anwendungsfälle}



\printbibliography[heading=subbibliography]
\end{refsection}

